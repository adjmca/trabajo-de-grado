% Capítulo 1

\chapter{El problema} % Título Principal del Capítulo

\label{Capitulo1} % Para hacer referenciar este capítudo use \ref{Capitulo1}

%----------------------------------------------------------------------------------------
%	SECTION 1
%----------------------------------------------------------------------------------------

\section{PLANTEAMIENTO DEL PROBLEMA.}

El estudio del tr\'ansito terrestre juega un papel fundamental cuando es requerido la remodelación o construcción de una vialidad, elevado o cualquier infraestructura ya que dicho estudio proporciona información de la cantidad de vehículos que circulan por una vía publica y así saber a ciencia cierta como sera el impacto generado en el tr\'ansito vehicular debido a lo mencionado anteriormente. El aforo vehicular esta vinculado directamente con el  EIV(ESTUDIO DEL IMPACTO VIAL) debido a que otorga un histograma de la cantidad de vehículos clasificados según el tipo que circulan cierta cantidad de horas al día durante un tiempo determinado, donde el mismo depende de la cantidad de días que se realice el conteo vehicular. Asimismo existe una cantidad considerable de usuarios que necesitan dicha información tales como las direcciones de tránsito y vialidad, compañías de ingeniería, constructoras, consultorías, empresas de mercadotecnia y cualquier otra empresa que requiera un aforo vehicular para un EIV, previo o posterior a algún proyecto.

Ahora bien el conteo de vehículos es realizado de manera manual lo que trae como consecuencia que se generen errores debido a las limitaciones del grupo de personas encargados del conteo, debido a ello se han realizado investigaciones en cuanto a la creación de algoritmos de conteo vehicular  así como también el diseño de programas que efectúen un conteo vehicular de forma digital, facilitando de esta manera el aforo vehicular y en consecuencia disminuyendo el error generado debido al conteo manual.

En cuanto a las herramientas que suelen ser utilizadas para solventar los inconvenientes
generados del conteo manual, la vídeo detección ofrece la mayor cantidad de ventajas, obteniendo mejor resultado en el reconocimiento vehicular; siendo las cámaras una de las herramientas más simples de instalar, ubicadas en la categoría de sensores no intrusivos, es decir, no afecta al tránsito durante su instalación o medición, permitiendo una detección óptima y una solución económica. En otras palabras esta es una herramienta versátil que le permite al usuario simplificar el trabajo y disminuir el tiempo de procesamiento impulsada por el avance tecnológico en el área de la visión por computador, además de contar vehículos, proporcionar datos como velocidad, tipo de vehículo, densidad, reconocimiento de placa, entre otros.

El trabajo de Quesada (2015) sobre un algoritmo de estimación del número de elementos móviles en vídeos digitales orientado a la gestión del tráfico vehicular, muestra un estudio sobre conteo vehicular utilizando vídeos con fondo estático de la base de datos Lankershin Boulevard, detectando a los vehículos y contándolos al pasar por una línea marcada por el programador. Los resultados fueron satisfactorios, disminuyendo el costo computacional, arrojando un error porcentual muy próximo a otros métodos actuales, generando el aporte de la implementación del algoritmo Principal Component Pursuit (PCP) y recomendaciones de gran importancia para futuros estudios.

En relación con el conteo vehicular de manera automatizada, en la universidad de Carabobo se realizo un sistema de aforo vehicular automatizado titulado "DESARROLLO DE UN SISTEMA DE AFORO VEHICULAR MEDIANTE PROCESAMIENTO DIGITAL DE VÍDEO" donde se plantearon algunas recomendaciones las cuales son plasmadas a continuación:

\begin{enumerate}
	\item Modificar el sistema desarrollado para ser utilizado en vídeos en vivo.
	\item Buscar otra forma de delimitar los intervalos de aforo y así distribuir en el tiempo correctamente los vehículos aforados, de manera que no se requiera de un vídeo con valores de cuadros por segundo constantes.
	\item Añadir al sistema un módulo de reconocimiento que sea capaz de monitorear, detectar y clasificar los resultados del módulo de detección con la finalidad de determinar si un vehículo corresponde a la clasificación de liviano, bus, pesado entre otros.
	\item Modificar el sistema desarrollado para disminuir el tiempo de aforo, implementando procesamiento en paralelo. De esta forma se podría analizar vídeos de larga duración en fracciones de su tiempo.
	
\end{enumerate}

Con esto queremos decir que partiremos de las recomendaciones planteadas en el trabajo de grado titulado "DESARROLLO DE UN SISTEMA DE AFORO VEHICULAR MEDIANTE PROCESAMIENTO DIGITAL DE VÍDEO" permitiendo realizar una actualización del mismo y por ende obtener un software mas potente. 
 

%-----------------------------------
%	SUBSECTION 1
%-----------------------------------
\section{JUSTIFICACIÓN DE LA INVESTIGACIÓN.}

El aforo vehicular manual usado en el EIV contiene algunas desventajas dentro de las cuales puede mencionarse el dinero usado en la contratación del personal encargado de efectuar el conteo vehicular, el error generado debido a la distracción por parte del personal, el tiempo invertido en dicho conteo; también debe ser tomado  en cuenta el hecho de que en el trabajo de grado titulado: DESARROLLO DE UN SISTEMA DE AFORO VEHICULAR MEDIANTE PROCESAMIENTO DIGITAL DE VÍDEO realizado en la FACULTAD DE INGENIERÍA DE LA UNIVERSIDAD DE CARABOBO dejo plasmadas unas recomendaciones dando a entender que puede realizarse una actualización del mismo creando de esta manera un software  m\'as poderoso. Es por ello que debe realizarse una actualización del SOFTWARE SISTEMA DE AFORO VEHICULAR MEDIANTE PROCESAMIENTO DIGITAL DE VÍDEO permitiendo así un EIV mas efectivo y fiable de manera automatizada, que tome en cuenta la clasificación del vehículo según su forma para así  determinar el grado de ocupación y las condiciones en que opera cada segmento de la vía p\'ublica, otro elemento importante a tomar en consideración es el tiempo usado en el conteo en donde el mismo debe ser lo mas corto posible; es por ello que se implementara un procesamiento en paralelo con lo que se podría procesar vídeos de larga duración en fracciones de su tiempo. 
%-----------------------------------
%SECTION 2
%-----------------------------------

\section{OBJETIVOS.}
\subsection{Objetivo General}
ACTUALIZACIÓN DEL SOFTWARE SISTEMA DE AFORO VEHICULAR MEDIANTE PROCESAMIENTO DIGITAL DE VÍDEO.
\subsection{Objetivos Específicos}
\begin{enumerate}
	\item Evaluar el software " SISTEMA DE AFORO VEHICULAR MEDIANTE PROCESAMIENTO DIGITAL DE VÍDEO ".
	\item Realizar las mejoras del Software para el conteo de vehículos automatizado.
	\item Evaluar y validar el software diseñado.
	
\end{enumerate}

\section{ALCANCES}
La ACTUALIZACIÓN DEL SOFTWARE SISTEMA DE AFORO VEHICULAR MEDIANTE PROCESAMIENTO DIGITAL DE VÍDEO permitirá realizar conteo de vehículos procesando vídeos en vivo, así como clasificar los automóviles según el tipo ya sea pesado, bus, liviano entre otros. El software diseñado sera multiplataforma y autoejecutable, la instalación del mismo sera intuitiva para cualquier usuario en general.

La versión mejorada del SOFTWARE SISTEMA DE AFORO VEHICULAR MEDIANTE PROCESAMIENTO DIGITAL DE VÍDEO podrá distribuir en el tiempo correctamente los vehículos aforados, de manera que no se requiera de un vídeo con valores de cuadros por segundo constantes. También tendrá la capacidad de analizar vídeos de larga duración en fracciones de su tiempo y así poder disminuir el tiempo de aforo debido al procesamiento en paralelo. Este software  sera usado por dos   entes gubernamentales, la Alcaldía de San Diego y la Alcaldía de Maracay para así constatar la calidad del mismo.