% Capítulo 3

\chapter{Procedimiento Metodológico} % Título Principal del Capítulo

\label{Capitulo3} % Para hacer referenciar este capítudo use \ref{Capitulo3}

%----------------------------------------------------------------------------------------
%	SECTION 1
%----------------------------------------------------------------------------------------

\section{Etapas de la investigación.}
La realización de este proyecto consta de 4 etapas las cuales serán descritas a continuación. 


%	SUBSECTION 1
\subsection{Fase 1: Evaluar el software "SISTEMA DE AFORO VEHICULAR MEDIANTE PROCESAMIENTO DIGITAL DE VÍDEO".}


Para empezar ya que se diseñara una versión mejorada de este software el objetivo principal de esta fase es evaluar el rendimiento, que tan llamativa e intuitiva es la interfaz gráfica, requerimientos; así como también si dicho software es multiplataforma.y así concluir si tomamos como punto de partida las lineas de código de este software o iniciamos desde cero.

\subsubsection{Actividad A.  Instalación del software}

Debe determinarse si el software es multiplataforma, si es fácil de instalar es decir si el mismo es un programa autoejecutable (instalación intuitiva para el usuario) o deben instalarse los paquetes que permitan que el software pueda ser instalado en un computador. Antes de la instalación deben  obtenerse los requerimientos mínimos del ordenador en el cual sera instalado el software; si los mismos no pueden obtenerse se procederá a su instalación en distintos ordenadores con distintas capacidades para así determinar cuales son los requerimientos mínimos y los óptimos en cuestión.

\subsubsection{Actividad B. Operabilidad del Software}

Se evaluar\'a que tan vistosa y perspicaz es la interfaz gráfica así como la rapidez de detección vehicular mediante procesamiento digital de vídeo. Los recursos del computador usados por el software deben ser tomados en consideración, es decir, si es poco o mucho en cuanto a la memoria RAM y el uso de la CPU; y así saber si la ejecución del mismo en una PC es fiable (el computador no este lento) o no para un usuario en general.


%-----------------------------------
%	SUBSECTION 2
%-----------------------------------
\subsection{Fase 2: Revisar las fuentes bibliográficas referentes al procesamiento de imágenes para la detección de objetos.}

Se realizar\'a un estudio de las distintas técnicas de detección de objetos así como los principales conceptos de procesamiento de imágenes.

\subsubsection{Actividad C. Recopilar información referente al procesamiento de imágenes.}

Se realizara una recopilación de documentos concernientes al procesamiento de la imagen a nivel computacional, ya sea en internet o en la biblioteca de la Universidad de Carabobo para así obtener conocimientos fuertes en dicha área.

\subsubsection{Actividad D. Estudio de detección de objetos.}

Se estudiaran los distintos algoritmos de detección de objetos, también serán realizados cursos en internet que hablen sobre la detección de objetos y de esta manera poder entender como es realizado la detección de objetos mediante el procesamiento de imágenes y como el ordenador aprende a partir de un modelo la dicha detección de un objeto en especifico. 

%-----------------------------------
%	SUBSECTION 3
%-----------------------------------

\subsection{Fase 3: Realizar  un estudio de algoritmos de conteo vehicular y la posterior selección del mas robusto.}
Se efectuara una evaluación de algoritmos de conteo vehicular para así determinar el de mejor rendimiento y de menor consumo de recursos por parte del ordenador.

\subsubsection{Actividad E. Evaluación de algoritmos de conteo vehicular.}

Debido al estudio previo de algoritmos de detección de objetos que permitió llevar a cabo la clasificación de los mismos respecto a sus capacidades de detección deberá buscarse el algoritmo mas idóneo para el conteo vehicular, aunque  sera evaluada la opción de tomar como algoritmo de conteo vehicular el usado por los estudiantes BLANCO BRITO OSCAR y HERNANDEZ ZAFRA EDWING en el trabajo de grado titulado "SISTEMA DE AFORO VEHICULAR MEDIANTE PROCESAMIENTO DIGITAL DE VÍDEO".


\subsection{Fase 4: Diseño del Software para el conteo de vehículos automatizado así como la evaluación y validación del mismo.}

En esta fase deben tomarse en cuenta los criterios de rediseño del software, ya sea el lenguaje de alto nivel que se usara si el mismo sera multiplataforma; fijar también los requerimientos de Hardware entre otras cosas la cuales serán descritas a continuación:

\subsubsection{Actividad F. Elección del lenguaje a usar en el diseño del software. }

Se efectuara una investigación de los distintos lenguajes de programación ya sea AdA, ALGOL, Basic, C++, Clipper, MATLAB, Java, COBOL, python, etc..., y así determinar cual es el mas versátil e idóneo para la creación del software;la escogencia del mismo esta sujeto a la cantidad de documentación que hable del lenguaje,que sea multiplataforma, orientado a objetos; entre otras cosas.

Por ultimo se deberá elegir que biblioteca de visión por computador se usara para poder realizar la detección y posterior conteo de los vehículos, la misma deberá estar actualizada y libre al publico en general.OpenCV y scikit-image pueden ser elegidas ya que ayudan a la finalidad principal del software, la detección y posterior clasificación según la forma del vehículo.

\subsubsection{Actividad G. Estructuración del software.}

En esta actividad deberán fijarse las características del software, ya sea la interfaz gráfica, los requerimientos mínimos de un ordenador para su plena operabilidad,si el mismo sera autoejecutable y multiplataforma así como el tiempo estimado de culminación de dicho software.

\subsection{Actividad H. Programación del software.}

En esta etapa del trabajo de grado se procede a la programación pertinente de las lineas de código usadas en la creación de la versión mejorada del SOFTWARE: SISTEMA DE AFORO VEHICULAR MEDIANTE PROCESAMIENTO DIGITAL DE VÍDEO . El camino dentro del cual se transitara en la actualización del software esta sujeto a ensayos y errores, cosa que es inevitable en el rediseño de un software. Serán tomados en cuenta los cursos disponibles, documentación en la web referente al lenguaje que se usara en la programación del software. Así como tutoriales prácticos sobre las técnicas de detección de objetos estudiadas con anterioridad. 

Ya para culminar solo queda recalcar que la realización de este trabajo de grado tomara en cuenta las recomendaciones planteadas por los ya ingenieros BLANCO BRITO OSCAR y HERNANDEZ ZAFRA EDWING en el trabajo de grado titulado "SISTEMA DE AFORO VEHICULAR MEDIANTE PROCESAMIENTO DIGITAL DE VÍDEO". También debe acotarse que puede que se usen las lineas de código del trabajo de grado antes mencionado en el diseño del software y así poder crear una versión mejorada del mismo
%----------------------------------------------------------------------------------------
%	SECTION 2
%----------------------------------------------------------------------------------------


\section{Cronograma de Actividades}

\newpage

\begin{figure}
\centering
%------------------------------------------------
%Definiciones Básicas (NO MODIFICAR)
%------------------------------------------------

\definecolor{barblue}{RGB}{153,204,254}
\definecolor{groupblue}{RGB}{51,102,254}
\definecolor{linkred}{RGB}{165,0,33}
%\renewcommand\sfdefault{phv}
\setganttlinklabel{s-s}{Inicio-Inicio}
\setganttlinklabel{f-s}{Inicio-Final}
\setganttlinklabel{f-f}{Final-Final}
\sffamily
\begin{ganttchart}[
%------------------------------------------------
%Encabezado de diagrama (NO MODIFICAR)
%------------------------------------------------
canvas/.append style={fill=none, draw=black!5, line width=.75pt},
hgrid style/.style={draw=black!5, line width=.75pt},
vgrid={*1{draw=black!5, line width=.75pt}},
%today=7,  % En este comando se indica el mes en el que se entrega el Proyecto. En caso de que ninguna actividad haya comenzado en el momento de la entrega, desactive (comente) esta línea.  
today rule/.style={
draw=black!64,
dash pattern=on 3.5pt off 4.5pt,
line width=1.5pt
},
today label font=\small\bfseries,
title/.style={draw=none, fill=none},
title label font=\bfseries\footnotesize,
title label node/.append style={below=7pt},
include title in canvas=false,
bar label font=\small\color{black!70},
bar label node/.append style={left=2cm},
bar/.append style={draw=none, fill=black!63},
bar incomplete/.append style={fill=barblue},
bar progress label font=\footnotesize\color{black!70},
group incomplete/.append style={fill=groupblue},
group left shift=0,
group right shift=0,
group height=.5,
group peaks tip position=0,
group label node/.append style={left=.6cm},
group progress label font=\bfseries\small,
link/.style={-latex, line width=1.5pt, linkred},
link label font=\scriptsize\bfseries,
link label node/.append style={below left=-2pt and 0pt}
]{1}{12}
%----------------------------------------------------------
%Fin del Encabezado de diagrama
%---------------------------------------------------------
% A partir de esta linea Ud podrá editar las lineas que aparezcan
%comentadas para adaptar el diagrama de gantt a sus requerimientos

\gantttitle[
title label node/.append style={below left=7pt and -3pt}
]{Meses:\quad1}{1}
\gantttitlelist{2,...,12}{1} \\ % Número de meses.
%Si desea mas especificidad en la fecha, modifique el comando anterior.
\ganttgroup[progress=10]{Fase 1}{1}{2} \\%Porcentaje
% de completitud del grupo de actividades.
\ganttbar[
progress=10,% Porcentaje de completitud de la actividad
name=WBS1A
]{Actividad A}{1}{2} \\%Coloque acá la primera 1ra actividad a realizar
% y su duración.
\ganttbar[
progress=0,% Porcentaje de completitud de la actividad
name=WBS1B
]{Actividad B}{1}{2} \\%Coloque acá la segunda 2da actividad a realizar
%y su duración.
\ganttgroup[progress=10]{Fase 2}{1}{4}\\
\ganttbar[
progress=0,% Porcentaje de completitud de la actividad
name=WBS1C
]{Actividad C}{2}{3} \\%Coloque acá la tercera 3ra actividad a realizar
%y su duración.
\ganttbar[
progress=0,% Porcentaje de completitud de la actividad
name=WBS1D
]{Actividad D}{2}{4} \\[grid]%Coloque acá la 4ta actividad actividad a
% realizar y su duración.
\ganttgroup[progress=0]{Fase 3}{4}{5} \\% Porcentaje 
%de completitud del grupo de actividades.

\ganttbar[progress=0,% Porcentaje de completitud de la actividad
name=WBS1E
]{Actividad E}{4}{5} \\%Coloque acá la 5ta actividad a realizar y su duración.


\ganttgroup[progress=0]{Fase 4}{5}{12}\\

\ganttbar[progress=0,% Porcentaje de completitud de la actividad
name=WBS1F
]{Actividad F}{5}{6} \\%Coloque acá la 6ta actividad a a realizar y su duración.
\ganttbar[progress=0,% Porcentaje de completitud de la actividad
name=WBS1G
]{Actividad G}{5}{6}\\%Coloque acá la primera 7ma actividad a realizar y su duración.
\ganttbar[progress=0,% Porcentaje de completitud de la actividad
name=WBS1H
]{Actividad H}{6}{12}\\%Coloque acá la primera 8va actividad a realizar y su duración.
\ganttlink[link type=f-s]{WBS1A}{WBS1B} % Esto permite enlazar dichas actividades con los conectores inicio-inicio (s-s), final-inicio (f-s) o final-final (f-f)
\ganttlink[link type=s-s]{WBS1B}{WBS1C}
\ganttlink[link type=f-s]{WBS1C}{WBS1D}
\ganttlink[link type=f-s]{WBS1D}{WBS1E}
\ganttlink[link type=f-s]{WBS1E}{WBS1F}
\ganttlink[link type=f-s]{WBS1F}{WBS1G}
\ganttlink[link type=f-s]{WBS1G}{WBS1H}
%\ganttlink[
%link type=f-f,
%link label node/.append style=left
%]{WBS1C}{WBS1D}
\end{ganttchart}

\caption{Diagrama de Gantt con indicadores y enlaces}
\label{GanttFig}
\end{figure}